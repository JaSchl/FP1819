% bei Standalone in documentclass noch:
% \RequirePackage{luatex85}

\documentclass[captions=tableheading, titlepage= firstiscover, parskip = half , bibliography=totoc]{scrartcl}
%paper = a5 fÃŒr andere optinen
% titlepage= firstiscover
% bibliography=totoc fÃŒr bibdateien
% parskip=half  VerÀnderung um AbsÀtze zu verbessern

\usepackage{scrhack} % nach \documentclass
\usepackage[aux]{rerunfilecheck}
\usepackage{polyglossia}
\usepackage[style=numeric, backend=biber]{biblatex} % mit [style = alphabetic oder numeric] nach polyglossia
\addbibresource{lit.bib}
\setmainlanguage{german}

\usepackage[autostyle]{csquotes}
\usepackage{amsmath} % unverzichtbare Mathe-Befehle
\usepackage{amssymb} % viele Mathe-Symbole
\usepackage{mathtools} % Erweiterungen fÃŒr amsmath
\usepackage{fontspec} % nach amssymb
% muss ins document: \usefonttheme{professionalfonts} % fÌr Beamer PrÀsentationen
\usepackage{longtable}
\usepackage{dsfont}
\usepackage[
math-style=ISO,    % \
bold-style=ISO,    % |
sans-style=italic, % | ISO-Standard folgen
nabla=upright,     % |
partial=upright,   % /
]{unicode-math} % "Does exactly what it says on the tin."
\setmathfont{Latin Modern Math}
% \setmathfont{Tex Gyre Pagella Math} % alternativ

\usepackage[
% die folgenden 3 nur einschalten bei documenten
locale=DE,
separate-uncertainty=true, % Immer Fehler mit ±
per-mode=symbol-or-fraction, % m/s im Text, sonst \frac
]{siunitx}

% alternativ:
% per-mode=reciprocal, % m s^{-1}
% output-decimal-marker=., % . statt , fÃŒr Dezimalzahlen

\usepackage[
version=4,
math-greek=default,
text-greek=default,
]{mhchem}

\usepackage[section, below]{placeins}
\usepackage{caption} % Captions schöner machen
\usepackage{graphicx}
\usepackage{grffile}
\usepackage{subcaption}

% \usepackage{showframe} Wenn man die Ramen sehen will

\usepackage{float}
\floatplacement{figure}{htbp}
\floatplacement{table}{htbp}

\usepackage{mathtools}

\usepackage{booktabs}

 \usepackage{microtype}
 \usepackage{xfrac}

 \usepackage{expl3}
 \usepackage{xparse}

 % \ExplSyntaxOn
 % \NewDocumentComman \I {}  %Befehl\I definieren, keine Argumente
 % {
 %    \symup{i}              %Ergebnis von \I
 % }
 % \ExplSyntaxOff

 \usepackage{pdflscape}
 \usepackage{mleftright}

 % Mit dem mathtools-Befehl \DeclarePairedDelimiter können Befehle erzeugen werden,
 % die Symbole um AusdrÃŒcke setzen.
 % \DeclarePairedDelimiter{\abs}{\lvert}{\rvert}
 % \DeclarePairedDelimiter{\norm}{\lVert}{\rVert}
 % in Mathe:
 %\abs{x} \abs*{\frac{1}{x}}
 %\norm{\symbf{y}}

 % FÃŒr Physik IV und Quantenmechanik
 \DeclarePairedDelimiter{\bra}{\langle}{\rvert}
 \DeclarePairedDelimiter{\ket}{\lvert}{\rangle}
 % <name> <#arguments> <left> <right> <body>
 \DeclarePairedDelimiterX{\braket}[2]{\langle}{\rangle}{
 #1 \delimsize| #2
 }

\setlength{\delimitershortfall}{-1sp}

 \usepackage{tikz}
 \usepackage{tikz-feynman}

 \usepackage{csvsimple}
 % Tabellen mit \csvautobooktabular{"file"}
 % muss in table umgebung gesetzt werden


% \multicolumn{#Spalten}{Ausrichtung}{Inhalt}

\usepackage{hyperref}
\usepackage{bookmark}
\usepackage[shortcuts]{extdash} %nach hyperref, bookmark

\newcommand{\ua}[1]{_\symup{#1}}
\newcommand{\su}[1]{\symup{#1}}

\begin{document}
\section{Results}
\subsection{Determination of the resonance points}
The resonance points can be taken from the resonance curves, which were drawn by the
XY-plotter. These are atteched in the appendix.
With the calibration of the x-axis every cm is linear propotional to 50\,mA, which is used for
the data of the table \ref{fig:Resonanzstelle} .
\newline
The values of the parallel orientation are placed in every first line of each frequency,
the values of the antiparallel orientation in the second.
\begin{table}
  \centering
  \caption{Distance and related currents of the resoncance points }
  \begin{tabular}{c c c c}
    \toprule
    {$\su{\nu_e}$\,/\,MHz}& $\su{x}\,/\,\mathrm{cm}$ & {I\,/\,$\mathrm{mA}$} & {B\,/\,$\su{\mu}$T} \\
    \midrule
     14,798 & 8,3 & 415 & 582 \\
            & 7,2 & 360 & 505 \\
     19,448 & 11,0 & 550 & 771 \\
            & 10,0 & 500 & 701 \\
     23,888 & 12,6 & 630 & 884 \\
            & 11,5 & 575 & 807 \\
     29,448 & 15,7 & 785 & 1101 \\
            & 14,5 & 725 & 1017 \\
    \bottomrule
    \label{fig:Resonanzstelle}
  \end{tabular}
\end{table}
\newline
The magnetic field is calculated with equation \ref{eqn:1}.

\subsection{Determinaton of the Landé g-factor and the earth's magnetic field}
The earths's magnetic field can be calculated with
\begin{equation*}
    B_{earth}=\frac{1}{2}(B_p-B_a).
\end{equation*}
In result the values of the magnetic field are listed in table \ref{fig:Erde}
\begin{table}
  \centering
  \caption{Values of the magnetic field}
  \begin{tabular}{c c}
    \toprule
    {$\su{\nu_e}$\,/\,MHz} & {$\su{B_{erd}}$\,/\,$\su{\mu}$T} \\
    \midrule
     14,798 & 38,5 \\
     19,448 & 35,0 \\
     23,888 & 38,5 \\
     29,448 & 42,0 \\
    \bottomrule
    \label{fig:Erde}
  \end{tabular}
\end{table}
\newline
The mean value, which can be estimated with
\begin{equation}
  \bar{x} = \frac{1}{n} \sum{x_n}
  \label{eqn:Mittelwert}
\end{equation}
and
\begin{equation}
\upsigma = \frac{1}{\sqrt{n}} \sqrt{\frac{\sum{(x_n - \bar{x})^2}}{n-1} },
\label{eqn:Standardabweichung}
\end{equation}
is
\begin{align*}
  \bar{B}_{earth} = 38,75\pm1,24.
\end{align*}
The Lande g-factor can be calculated with
\begin{equation*}
    g  = \frac{h\nu_e}{B\mu_b}.
\end{equation*}
In result the eight g-factors for each frequency and orientation are listed
in table \ref{fig:Lande}
\begin{table}
  \centering
  \caption{Values of the g-factor}
  \begin{tabular}{c c c}
    \toprule
    {$\su{\nu_e}$\,/\,MHz}& $\su{x}\,/\,\mathrm{cm}$ & {g} \\
    \midrule
     14,798 & 8,3 & 1,82 \\
            & 7,2 & 2,09 \\
     19,448 & 11,0 & 1,80 \\
            & 10,0 & 1,98 \\
     23,888 & 12,6 & 1,93 \\
            & 11,5 & 2,12 \\
     29,448 & 15,7 & 1,59 \\
            & 14,5 & 1,72 \\
    \bottomrule
    \label{fig:Lande}
  \end{tabular}
\end{table}
\newline
The mean value of the g-factor is calculated with \ref{eqn:Mittelwert} and \ref{eqn:Standardabweichung}.
It amounts of
\begin{align*}
    \bar{g} = 1,88\pm0,06.
\end{align*}

\newpage
\section{Discussion}
All of the results are presented and compared to their theoretical value in table \ref{fig:theo}.
\begin{table}
  \centering
  \caption{Comparison between measures and theoretical value}
  \begin{tabular}{c c c c}
    \toprule
     & $\text{Measures}$ & $\text{Theoretical value} $& $\text{Deviation in percent}$\\
    \midrule
     $\su{B_{earth}}$ & 38,75\pm1,24 $\su{\mu}$T & 44 \cite{1} $\su{\mu}$T & 11,93 \% \\
     g & 1,88\pm0,06 & 1,761 \cite{2} & 6,76 \% \\
    \bottomrule
    \label{fig:theo}
  \end{tabular}
\end{table}
\newline
Eventhough the deviation in percent is with 6,67\% for the g-factor and 11,93\% for the
earth's magnetic field small, there are some sources of error in the implentimantation aswell as in the analysis
of the resonance curves. The adjustment of the bridge voltage was imprecised because it was not possible
to regulate the voltage close to zero. At some point the resonance curve was hard to find and additionally
the drawing with the XY-plotter made the resonance point even more flatter. By analysing the curves by hand
it was, especially for the 23,888\,MHz frequency, difficult to get exact values of the coordinate paper.
\newline
Nevertheless the generated results are close to the theoretical values despite of some inaccuracies in the
implentimantation and analysis by hand.
\end{document}
