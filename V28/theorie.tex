% bei Standalone in documentclass noch:
% \RequirePackage{luatex85}

\documentclass[captions=tableheading, titlepage= firstiscover, parskip = half , bibliography=totoc]{scrartcl}
%paper = a5 fÃŒr andere optinen
% titlepage= firstiscover
% bibliography=totoc fÃŒr bibdateien
% parskip=half  VerÀnderung um AbsÀtze zu verbessern

\usepackage{scrhack} % nach \documentclass
\usepackage[aux]{rerunfilecheck}
\usepackage{polyglossia}
\usepackage[style=numeric, backend=biber]{biblatex} % mit [style = alphabetic oder numeric] nach polyglossia
\addbibresource{lit.bib}
\setmainlanguage{german}

\usepackage[autostyle]{csquotes}
\usepackage{amsmath} % unverzichtbare Mathe-Befehle
\usepackage{amssymb} % viele Mathe-Symbole
\usepackage{mathtools} % Erweiterungen fÃŒr amsmath
\usepackage{fontspec} % nach amssymb
% muss ins document: \usefonttheme{professionalfonts} % fÌr Beamer PrÀsentationen
\usepackage{longtable}
\usepackage{dsfont}
\usepackage[
math-style=ISO,    % \
bold-style=ISO,    % |
sans-style=italic, % | ISO-Standard folgen
nabla=upright,     % |
partial=upright,   % /
]{unicode-math} % "Does exactly what it says on the tin."
\setmathfont{Latin Modern Math}
% \setmathfont{Tex Gyre Pagella Math} % alternativ

\usepackage[
% die folgenden 3 nur einschalten bei documenten
locale=DE,
separate-uncertainty=true, % Immer Fehler mit ±
per-mode=symbol-or-fraction, % m/s im Text, sonst \frac
]{siunitx}

% alternativ:
% per-mode=reciprocal, % m s^{-1}
% output-decimal-marker=., % . statt , fÃŒr Dezimalzahlen

\usepackage[
version=4,
math-greek=default,
text-greek=default,
]{mhchem}

\usepackage[section, below]{placeins}
\usepackage{caption} % Captions schöner machen
\usepackage{graphicx}
\usepackage{grffile}
\usepackage{subcaption}

% \usepackage{showframe} Wenn man die Ramen sehen will

\usepackage{float}
\floatplacement{figure}{htbp}
\floatplacement{table}{htbp}

\usepackage{mathtools}

\usepackage{booktabs}

 \usepackage{microtype}
 \usepackage{xfrac}

 \usepackage{expl3}
 \usepackage{xparse}

 % \ExplSyntaxOn
 % \NewDocumentComman \I {}  %Befehl\I definieren, keine Argumente
 % {
 %    \symup{i}              %Ergebnis von \I
 % }
 % \ExplSyntaxOff

 \usepackage{pdflscape}
 \usepackage{mleftright}

 % Mit dem mathtools-Befehl \DeclarePairedDelimiter können Befehle erzeugen werden,
 % die Symbole um AusdrÃŒcke setzen.
 % \DeclarePairedDelimiter{\abs}{\lvert}{\rvert}
 % \DeclarePairedDelimiter{\norm}{\lVert}{\rVert}
 % in Mathe:
 %\abs{x} \abs*{\frac{1}{x}}
 %\norm{\symbf{y}}

 % FÃŒr Physik IV und Quantenmechanik
 \DeclarePairedDelimiter{\bra}{\langle}{\rvert}
 \DeclarePairedDelimiter{\ket}{\lvert}{\rangle}
 % <name> <#arguments> <left> <right> <body>
 \DeclarePairedDelimiterX{\braket}[2]{\langle}{\rangle}{
 #1 \delimsize| #2
 }

\setlength{\delimitershortfall}{-1sp}

 \usepackage{tikz}
 \usepackage{tikz-feynman}

 \usepackage{csvsimple}
 % Tabellen mit \csvautobooktabular{"file"}
 % muss in table umgebung gesetzt werden


% \multicolumn{#Spalten}{Ausrichtung}{Inhalt}

\usepackage{hyperref}
\usepackage{bookmark}
\usepackage[shortcuts]{extdash} %nach hyperref, bookmark

\newcommand{\ua}[1]{_\symup{#1}}
\newcommand{\su}[1]{\symup{#1}}

\begin{document}
\section{Zielsetzung}

The purpose of the experiment is to determine the magnetic moment of a free electron.
To implement this the electronspin-resonance is used.


\section{Theorie}
In quantum mechanics the status of particles is described by a wave function.
\begin{equation}
  \psi_{n,l,m}(r,\theta,\phi) = R_{n,l}(r) \cdot \Theta_{l,m}(\theta) \cdot \Phi_m(\phi)
\end{equation}
R is the radial part, $\Theta$ discribes the polar angle and
$\Phi$ is the part of the azimuth angle.
n, l, m are quantum numbers.
n is the main number and describes the niveaus of energy.
l is the number of orbital angular momentum.
m characterise the orientation of the system and can take the numbers 2l+1.

The current density defined by
\begin{equation}
  S = \frac{\hbar}{2im_0}\cdot (\psi* \nabla \psi - \psi \nabla \psi* )
\end{equation}
It results of moving electrons on the shell of an atom.
The resulting magnetic field have a magentic moment
\begin{equation*}
  \mu_z =\mu_b \cdot m
\end{equation*}
$\mu_b$ is the Bohr magneton.
Figure \ref{fig:muz} shows the geometric thoughts to deduce the magentic moment.

In a homogen magnetic field is the magnetic momentum of the electron shell
connected to the outer magnetic field.
The quantization of the direction leads to the zeeman-effect.
It describes the split of the energy niveau in teh outa magnetic field.
It is shown in figure \ref{fig:eniveau}
A split of an electron without orbital angular momentum is not expectet.
But it happens a split into two niveaus.
So the electron has to have a spin.
The electron is a fermion and have the spin  $|S| = \sfrac{1}{2}$.
The z-component of the magnetic moment can take two directions wich are determined by
the quantum number $m_s \in [-1/2, 1/2]$ .
\begin{equation*}
  \mu_{sz} = -gm_s\mu_B
\end{equation*}
The energy difference between two niveaus is
\begin{equation}
  \Delta E = g\mu_B B
\end{equation}
g is the gyromagnetic relation or Lande- Factor.

In thermal equilibrium the occupation of the two energy niveaus is described
by the Boltzmann-statistic. The lower niveau is on a higher level than the upper.
The Energy $\Delta E$ is given to the system by inserting high frequenzy HF-quantum.
The Spin of the electrons is turning over into a higher state.
The changing of the spin changes the sample magnetization.
It can be read out by a bridge circuit. 


\section{Durchführung}
\end{document}
