% bei Standalone in documentclass noch:
% \RequirePackage{luatex85}

\documentclass[captions=tableheading, titlepage= firstiscover, parskip = half , bibliography=totoc]{scrartcl}
%paper = a5 fÃŒr andere optinen
% titlepage= firstiscover
% bibliography=totoc fÃŒr bibdateien
% parskip=half  VerÀnderung um AbsÀtze zu verbessern

\usepackage{scrhack} % nach \documentclass
\usepackage[aux]{rerunfilecheck}
\usepackage{polyglossia}
\usepackage[style=numeric, backend=biber]{biblatex} % mit [style = alphabetic oder numeric] nach polyglossia
\addbibresource{lit.bib}
\setmainlanguage{german}

\usepackage[autostyle]{csquotes}
\usepackage{amsmath} % unverzichtbare Mathe-Befehle
\usepackage{amssymb} % viele Mathe-Symbole
\usepackage{mathtools} % Erweiterungen fÃŒr amsmath
\usepackage{fontspec} % nach amssymb
% muss ins document: \usefonttheme{professionalfonts} % fÌr Beamer PrÀsentationen
\usepackage{longtable}
\usepackage{dsfont}
\usepackage[
math-style=ISO,    % \
bold-style=ISO,    % |
sans-style=italic, % | ISO-Standard folgen
nabla=upright,     % |
partial=upright,   % /
]{unicode-math} % "Does exactly what it says on the tin."
\setmathfont{Latin Modern Math}
% \setmathfont{Tex Gyre Pagella Math} % alternativ

\usepackage[
% die folgenden 3 nur einschalten bei documenten
locale=DE,
separate-uncertainty=true, % Immer Fehler mit ±
per-mode=symbol-or-fraction, % m/s im Text, sonst \frac
]{siunitx}

% alternativ:
% per-mode=reciprocal, % m s^{-1}
% output-decimal-marker=., % . statt , fÃŒr Dezimalzahlen

\usepackage[
version=4,
math-greek=default,
text-greek=default,
]{mhchem}

\usepackage[section, below]{placeins}
\usepackage{caption} % Captions schöner machen
\usepackage{graphicx}
\usepackage{grffile}
\usepackage{subcaption}

% \usepackage{showframe} Wenn man die Ramen sehen will

\usepackage{float}
\floatplacement{figure}{htbp}
\floatplacement{table}{htbp}

\usepackage{mathtools}

\usepackage{booktabs}

 \usepackage{microtype}
 \usepackage{xfrac}

 \usepackage{expl3}
 \usepackage{xparse}

 % \ExplSyntaxOn
 % \NewDocumentComman \I {}  %Befehl\I definieren, keine Argumente
 % {
 %    \symup{i}              %Ergebnis von \I
 % }
 % \ExplSyntaxOff

 \usepackage{pdflscape}
 \usepackage{mleftright}

 % Mit dem mathtools-Befehl \DeclarePairedDelimiter können Befehle erzeugen werden,
 % die Symbole um AusdrÃŒcke setzen.
 % \DeclarePairedDelimiter{\abs}{\lvert}{\rvert}
 % \DeclarePairedDelimiter{\norm}{\lVert}{\rVert}
 % in Mathe:
 %\abs{x} \abs*{\frac{1}{x}}
 %\norm{\symbf{y}}

 % FÃŒr Physik IV und Quantenmechanik
 \DeclarePairedDelimiter{\bra}{\langle}{\rvert}
 \DeclarePairedDelimiter{\ket}{\lvert}{\rangle}
 % <name> <#arguments> <left> <right> <body>
 \DeclarePairedDelimiterX{\braket}[2]{\langle}{\rangle}{
 #1 \delimsize| #2
 }

\setlength{\delimitershortfall}{-1sp}

 \usepackage{tikz}
 \usepackage{tikz-feynman}

 \usepackage{csvsimple}
 % Tabellen mit \csvautobooktabular{"file"}
 % muss in table umgebung gesetzt werden


% \multicolumn{#Spalten}{Ausrichtung}{Inhalt}

\usepackage{hyperref}
\usepackage{bookmark}
\usepackage[shortcuts]{extdash} %nach hyperref, bookmark

\newcommand{\ua}[1]{_\symup{#1}}
\newcommand{\su}[1]{\symup{#1}}

\usepackage{amssymb}
\begin{document}
\section{Auswertung}
\subsection{Untersuchung eines Reflexklytrons}
\begin{table}
  \centering
  \begin{tabular}{c c c c c c}
    \toprule
    {Mode} & {$\su{U_0}$\,/\,V} & {$\su{U_1}$\,/\,V} & {$\su{U_2}$\,/\,V} & {$\su{A_0}$\,/\,V} & {$\su{f_0}$\,/\,MHz} \\
    \midrule
    1 & 60 & 50 & 70 & 0,9 & 9015 \\
    2 & 100 & 90 & 110 & 1,3 & 9007 \\
    3 & 150 & 140 & 170 & 2,5 & 9001 \\

    \bottomrule
    \label{fig:reflex}
  \end{tabular}
  \caption{Messwerte für die Reflektorspannungen mit der jeweiligen Amplitude und Frequenz}
\end{table}
Aus den Messwerten kann eine Regression der Form
\begin{align*}
    A(U) = xU^2 + yU+ z
\end{align*}
durchgeführt werden. Diese ist in Abbildung \ref{fig:regression} zu sehen.


Die daraus resultierenden Parameter sind in Tabelle \ref{tab:params} aufgetragen.
\begin{table}
    \centering
    \begin{tabular}{c c c}
        \toprule
        {x} & {y} & {z}\\
        \midrule

        \bottomrule
        \label{fig:params}
    \end{tabular}
    \caption{}
\end{table}


\begin{table}
    \centering
    \begin{tabular}{c c}
        \toprule
        {$\su{U_0}$\,/\,V} & {$\su{f_0}$\,/\,MHz} \\
        \midrule
         235 & 9001 \\
         230 & 8985 \\
         250 & 9021 \\
        \bottomrule
        \label{fig:}
    \end{tabular}
    \caption{}
\end{table}
Die Bandbreite $\Delta f$ berechnet sich nach
\begin{align*}
    \Delta f = f^{\shortmid} -f^{\shortmid\shortmid} = 15\,\mathrm{MHz}.
\end{align*}
\newline
Für die Abstimmempfindlichkteit ergibt sich mithilfe der zuvor berechneten Bandbreite
\begin{align*}
    E = \frac{f^{\shortmid} - f^{\shortmid\shortmid}}{U^{\shortmid} - U^{\shortmid\shortmid}} = 0,42 \,\frac{\su{MHz}}{\su{V}}.
\end{align*}

\newpage
\subsection{Messung von Frequenz, Wellenlänge und Dämpfung}
Die Messwerte sind in Tabelle \ref{tab:frequenz} dargestellt.
\begin{table}
    \centering
    \begin{tabular}{c c c c c}
        \toprule
        {$\su{f}$\,/\,MHz} & {1. Minimum\,/\,mm} & {2. Minimum\,/\,mm} & {$\su{\lambda_g}$\,/\,mm} & {a\,/\,mm}\\
        \midrule
        8952 & 72,6 & 97,5 & 43,7 & 21,85\\
        \bottomrule
        \label{fig:frequenz}
    \end{tabular}
    \caption{}
\end{table}
Der Abstand a ist dabei genau der Abstand zwischen den beiden Minima.
Aus diesem wird die Hohlwellenlänge $\lambda_g$ berechnet, welche genau dem doppelte Abstand a entspricht.
\newline
Mit Formel \ref{eqn:1} ergibt sich die Frequenz $\su{f} = 9701,84$\,MHz.
\newline
In Tabelle \ref{tab:swr} sind die Messergebnisse für die Dämpfung zu sehen.
Bei der Mikrometereinstellung sind die Werte auf 2,5\,mm normiert.
\begin{table}
    \centering
    \begin{tabular}{c c c}
        \toprule
        {SWR-Meter Ausschlag\,/\,dB} & {Mikrometereinstellung\,/\,mm} & {Dämpfung aus der Eichkurve\,/\,dB} \\
        \midrule
         0 & 0,00 & 0,0 \\
         2 & 0,50 & 1,0 \\
         4 & 0,70 & 1,5 \\
         6 & 1,00 & 2,0 \\
         8 & 1,15 & 3,0 \\
         10 & 1,25 & 4,5 \\
        \bottomrule
        \label{tab:swr}
    \end{tabular}
    \caption{}
\end{table}
In Abbildung \ref{fig:dämpfung} ist die Dämpfungskurve dargestellt.
\newpage
\subsection{Stehwellenmessung}
\begin{table}
    \centering
    \begin{tabular}{c c c}
        \toprule
        {Sondentiefe\,/\,V} & {SWR-Meter Ausschlag} \\
        \midrule
         3 & 1,430 \\
         5 & 1,625 \\
         7 & 2,400 \\
         9 & 4,000 \\
        \bottomrule
        \label{fig:}
    \end{tabular}
    \caption{SWR-Meter-Methode}
\end{table}

\begin{table}
    \centering
    \begin{tabular}{c c c c c c}
        \toprule
        {$\su{d_1}$\,/\,mm} & {$\su{d_2}$\,/\,mm} & {1. Minimum\,/\,mm} & {2. Minimum\,/\,mm} & {$\su{\lambda_g}$} & {SWR}\\
        \midrule
         77 & 54 & 69 & 72 & 27,2 & 4,88 \\
        \bottomrule
        \label{fig:}
    \end{tabular}
    \caption{3\,dB-Methode}
\end{table}

\begin{table}
    \centering
    \begin{tabular}{c c c c}
        \toprule
        {$\su{A_1}$\,/\,dB} & {$\su{A_2}$\,/\,dB} & {$\su{|A_2-A_1|}$\,/\,dB} & {SWR} \\
        \midrule
        20 & 27,14 & 7,14 & 3,57 \\
        \bottomrule
        \label{fig:abschwächer}
    \end{tabular}
    \caption{Abschwächer-Methode}
\end{table}




\end{document}
